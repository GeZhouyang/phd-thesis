\noindent%
Two studies of two-dimensional models of flows influenced by stratification and stratification/rotation are carried out in
order to investigate whether a two-dimensional model can reproduce a downscale energy cascade with an associated $ k^{-5/3} $ wavenumber  spectrum.
 Firstly, a series of highly resolved numerical simulations of the classical shallow water model is carried out. A forward energy cascade is observed but the dynamics is dominated by shocks, with an associated $ k^{-2} $-spectrum.
A theory for shallow water wave turbulence is formulated and compared to the results from the simulations.
 Secondly, a series of simulations of a new two-dimensional toy model is carried out, showing that  the model is not generating  shocks and can reproduce a downscale energy cascade with
an associated $ k^{-5/3} $ spectrum. The energy transfer is studied in detail in Fourier space and is compared with results from a general circulation model.

An experimental study of strongly stratified turbulence at the Coriolis platform in Grenoble is carried out, with the aim of testing novel theories of stratified turbulence. Turbulence is generated by traversing an array of cylinders through a tank containing stratified salt water. Velocity is measured by Particle Image Velocimetry (PIV) and density is measured by conductivity probes. In particular, the author has developed the software system analysing the PIV images. Preliminary results from the experiment are presented.

 To realise the research objectives, a set of open-source software packages are developed in Python, under the
umbrella of the FluidDyn project. The
packages enable execution of simulations, experiments and processing of data.
The codes are well documented, tested and designed to
promote development and reuse.

\keywords{geophysical flows, shallow water wave turbulence, energy cascade,
stratified turbulence, waves and vortices, open source software}
