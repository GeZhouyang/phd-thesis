%% Journal of Open Research Software Latex template -- Created By Stephen
%% Bonner and John Brennan, Durham Universtiy, UK.
%% see http://openresearchsoftware.metajnl.com

\chapter{Towards reproducible, open science through open research software}
%TODO: Fix table of contents
\section{Introduction}

% \vspace{0.5cm}

% \section*{Title}

% \textcolor{blue}{The title of the software paper should focus on the software,
% e.g. “Text mining software from the X project”. If the software is closely
% linked to a specific research paper, then “Software from Paper Title” is
% appropriate. The title should be factual, relating to the functionality of the
% software and the area it relates to rather than making claims about the
% software, e.g. “Easy-to-use”.}

% old title

% FluidDyn: a Python open-source framework for research and teaching in fluid
% dynamics

% pa: this title may not be really good for this paper...
% - "research and teaching" may not be very important in the title
% - Recent progress in open-source software engineering and consequences on
%   science

% todo: find a better title!

% Taking advantage of recent advances in open-source software engineering to study
% fluid dynamics: the example of the FluidDyn project


% \textcolor{blue}{An overview of the software, how it was produced, and the
% research for which it has been used, including references to relevant research
% articles. A short comparison with software which implements similar
% functionality should be included in this section. }

\subsection{Science, software, open-source and the computer revolution}

Science is mainly a collective activity.  We can go further only by
\href{https://en.wikipedia.org/wiki/%
Standing_on_the_shoulders_of_giants}{``standing on the shoulders of giants''}
(and of a huge number of technicians and scientists).  Science is a lot about
how to build new knowledge from the work of others and hence, the exchange
of ideas is a fundamental aspect.
%
In the last decades, we have lived through a revolution on how people exchange
ideas.
%
Computers of all kinds (from smartphones to HPC clusters) connected by a world
wide web are used for human communication and also for many other applications.
%
It has become nearly effortless to reproduce and exchange ideas and data.
%
The set of intangibles that grows when shared and degrades when hoarded, such as
knowledge and love, has been somehow extended.
% Relevant quote: 
% O Goddess Saraswati, your Treasure is unique in nature.  It increases when it
% is freely spent and gets destroyed if it is hoarded (and not shared with
% others).
%
The information technology revolution open doors to fantastic opportunities for
human collaboration, and of course, for science.

Web related activities alone account for at least 5\% of GDP in the USA and the
European Union\footnote{See the report by Internet Association titled
\href{https://internetassociation.org/reports/refreshing-understanding-internet-economy-ia-report/}
{``Refreshing Our Understanding of the Internet Economy''}}.
% \cite{hooton_christopher_refreshing_2017}.
A huge amount of money (and work) is invested on the related technologies. We
are familiar with the most prominent companies involved in this dynamics
(Google, Facebook, etc.), but there are also several smaller and sometimes
lesser known organizations.  Most of these companies base part of their work on
the open-source paradigm contributing to open-source languages, libraries,
software and operating systems, while also using them --- a win-win situation
for both the corporations and the community.
%
This has lead to deep changes in software engineering, with a massive use of
open-source methods and tools, for example
\href{https://en.wikipedia.org/wiki/Distributed_version_control}{distributed
version control systems (DVCS)} and web-based source development platforms.

The computer performance continues to
\href{https://en.wikipedia.org/wiki/Moore\%27s_law}{increase exponentially},
now also with the help of Graphical Processing Units (GPU). This gave way to a
big boom in practical uses of data science and machine learning, which drives a
strong research on artificial intelligence.  This dynamics contributes to
progresses in open-source software.

These changes in our world also reflect in the way science is done.
%
Software and programming in science occupy a much bigger place than before.
%
% see http://sciencecodemanifesto.org/
% http://lorenabarba.com/gallery/reproducibility-pi-manifesto/
% https://www.numfocus.org/
%
The role of software in science has changed. In the past, coding was sometimes
considered as an inferior activity by some scientists.  The focus was on the
theory and the mathematical demonstration, which had to be elegant as it gets
included in the articles.  In contrast, it was normal to write crude code
% av: crappy = shitty!
% alternatives: shabby, inelegant, amateurish, unpolished, clumsy
% https://en.oxforddictionaries.com/thesaurus/amateurish
and to just show the results. Nowadays, codes tend to be at the heart of
research.

``Open-science'' is a new trend taking advantage of these new facts.  Pioneering
attempts are being made to do better science, improving reproducibility and
collective efficiency, by using the open-source methods and tools for science
and sharing and collaborating via the world wide web.

% http://www.nature.com/news/interactive-notebooks-sharing-the-code-1.16261

This paper presents FluidDyn, a project to foster open-science and open-source in
fluid mechanics. This first section ``Overview'' is organized into two large
parts.  A long introduction presents the context and describes our motivations and
the concept behind the project FluidDyn.  It is organized as follows:
\begin{itemize}
\item A scientific project based on the collaboration through open-source, such
as FluidDyn, is only possible thanks to the recent developments in methods and
tools for open-source software engineering. We start by presenting these
developments and discuss the implications for the scientists.

\item FluidDyn is based on the language Python. Incidently, it is the defining
characteristics of Python which make the eventual success of such a
project realizable.

Python is one of the most important tool in recent open-source dynamics and in
particular in science.
%
Python in fluid mechanics is growing fast, but is yet to reach widespread
adoption as the language of choice. Therefore, it is useful to present its
characteristics for readers who are unacquainted with its nuances.

\item We will discuss about the possible contradiction between productivity of
individuals and productivity at the community level.

\item We will establish a picture of software engineering in fluid mechanics
and present some of the main tools commonly used.

\item The introduction finally ends with a summary of the motivations for the
project FluidDyn and a presentation of its principles.
\end{itemize}

%TODO: Fix names
The section ``Introduction'' is completed by a second subsection called
``Implementation and architecture'' devoted to present the organization of the
packages of the project and to describe in more detail the base package
\fluiddyn\footnote{We use FluidDyn (with capital letters) to name the project and
\fluiddyn for the base package.}.

\subsection{Methods and tools for open-source software engineering}

\paragraph{Free and Open-Source Software (FOSS).} The term \emph{free} in
FOSS is a misnomer, as it actually stands for \emph{freedom} (to use, modify
and distribute). The FOSS movement has dramatically decreased the cost of using
computers, and is evident from the widespread use of GNU/Linux systems in
desktops, computing clusters and web-servers in academia and beyond.
%
The beginnings of FOSS culture can be traced back to the success for scientific
publication of the free type-setting standards \TeX\ (1977) and \LaTeX\ (1985).
%
Another founding moment for the FOSS movement was the launch of the GNU project
in 1983 by \href{https://en.wikipedia.org/wiki/Richard_Stallman}{Richard
Stallman}, to create a Unix-like computer operating system composed entirely of
free software\footnote{For completeness, see also the work done on
\href{https://www.levenez.com/unix/}{other Unix operating systems,
for example BSD}.}.
%
GNU is today known for its compiler collection (GCC) and a multitude of tools
which when combined with the Linux kernel (created by
\href{https://en.wikipedia.org/wiki/Linus_Torvalds}{Linus Torvalds} in 1991)
forms the GNU/Linux operating systems that we are familiar today.
%
FOSS movement has had huge successes in many frontiers --- Apache could be
termed as the ``first killer-app of Linux'' and now with an ever-increasing
suite of softwares including Firefox, LibreOffice, Gimp as solid alternatives
to proprietary offerings.
%
Linux has become the most widely used kernel, being deployed on servers,
personal computers, embedded devices, and also smart phones (with Android).

% remark Julien Salort: not interesting. Nothing on BSD.
% https://www.levenez.com/unix/

Over the years, FOSS development has transitioned from an organic community of
volunteers, towards an organized system with participation from industries,
non-profit organizations and government institutions. This could be regarded as
the second-age of FOSS movement~\cite{fitzgerald_transformation_2006}.

%todo: DONE
% Here, we have to add a paragraph on free and open-source (libre)
% software:

% The following facts can be mentioned:

% - tex (started in 1977), latex (1985)

% - GNU (started 1983), FSF in 1985, licence and philosophy...
% first complete system (with the Linux kernel) in 1991

% - 90s arrival of internet (soon mass market)

% Some huge open-source successes : Apache ("first killer app of Linux"),
% now Firefox, Open-office
% av: https://www.reddit.com/r/AskReddit/comments/7x639l/what_free_software_is_so_good_you_cant_believe/du6pw11/

% Linux kernel now widely used on servers, personal computers, embedded
% devices, smarth phone (with Android).
% https://en.wikipedia.org/wiki/Linux_kernel

% Now, new period for open-source: 
% see http://www.cepis.org/upgrade/files/full-2005-III.pdf

% - "Libre Software Movement: The Next Evolution of The IT Production
% Organization?"

% - ``the composition of development teams was changing, from all-volunteer
% teams to teams with paid participants from industry, government or
% not-for-profit organizations.''

% By the way, we also have to use the term "libre software" (?)
% av: Libre-software is not so mainstream. Even Linux kernel is not libre with
% binary blobs for drivers.

% We may have to change the order of this list?
% - git and mercurial 2005
% - software repositories ~ 10 years before
% so I would exchange the too paragraph...
% av:That is probably because we did not mention patches and CVS, subversion
% etc.


\paragraph{Distributed Version Control Systems (DVCS) and web-based source
development platforms.}  Collaboration was achieved in the early years of FOSS
development through emailing ``patches'', and centralized VCS near the turn of
the 21\textsuperscript{st} century.  The collective collaboration on
development is today streamlined with the emergence of DVCS (especially, Git
and Mercurial) and web-based source development platforms like GitHub,
Bitbucket and more recently GitLab.
%
These tools are designed to work on any kind of text files, not just codes, so
they could even be utilized to write scientific documents with multiple authors.
%
These tools are so useful for work, that using DVCS in science shall reap
benefits in terms of productivity in the long run~\cite{wilson_best_2014}.

Git has become the \textit{de facto} standard for source management. However,
for the FluidDyn project, we prefer to use Mercurial because we consider it to
be simpler for the beginners and just as powerful for expert users\footnote{See
\href{https://www.mercurial-scm.org/wiki/GitConcepts}{``a detailed
comparison''} on Mercurial website.}. Simplicity and lowering the learning
barrier for scientists who are not experts in software development is crucial
for our project.
%
Note that, it is now possible to use Mercurial to work with Git repositories,
for example hosted in GitHub and GitLab\footnote{By using the Mercurial
extension \href{http://hg-git.github.io/}{hg-git}.}.

\paragraph{Package management and third-party software repositories.} Python
as a medium for scientific computing owes its success to a rich ecosystem of
third-party packages.  Smooth deployment of these packages requires a package
manager and a supported repository. A credible hub to upload source codes as
compressed archives or as pre-compiled binary packages was therefore
indispensable. The use of repositories started with the conception of CPAN and
CRAN repositories for the languages Perl and R in 1993 --- based on the
Comprehensive TeX Archive Network (CTAN) model for \TeX\ packages. 

Today, \href{https://pypi.org}{Python Package Index (PyPI)} and
\href{https://anaconda.org}{Anaconda Cloud} are major repositories for
delivering and downloading Python packages, assisted by package managers
\pack{pip} and \pack{conda} respectively.
% Almost all major 
% languages\footnote{\href{https://stackoverflow.com/questions/1693529/%
% list-of-top-repositories-by-programming-language}{https://stackoverflow.com/%
% questions/1693529/list-of-top-repositories-by-programming-language}}
% have similar repositories, with exceptions such as Fortran and Matlab.
Specialized package managers have also emerged to install scientific
libraries and compilers such as \pack{easybuild} and \pack{spack}.

\paragraph{Continuous integration (CI).} The need for strong reliability of
programs with increasing complexity has forced developers to use continuous
integration.
%
In CI, firstly the code is \emph{built} against a known, repeatable emulation
environment on a server. Thereafter, a set of \emph{unit tests} are run for each
commit (modification of the code) to check that no bugs have been introduced. It
is also important to analyse what portion of the source code is covered by these
unit tests --- a method referred to as \emph{code coverage}. In FluidDyn project
we stop at this level, but it is possible to further enhance it by adding code
style checks by running linters alongside unit tests; and also to add
\emph{continuous delivery} of documentation and package releases.

Almost all CI platforms are free for open-source projects --- Travis CI,
CircleCI and Appveyor % (sic) av: I have double checked the official names
to name a few. Bitbucket and GitLab have built-in CI
implementations.  Websites such as Codecov and Coveralls help to chart and
analyse the code coverage history, viewable by anyone.

When the code depends on a wide variety of external dependencies, CI is useful
to avoid \href{https://en.wikipedia.org/wiki/Software\_rot}{software rot},
i.e.\ to avoid becoming obsolete with newer versions of the compiler or its
dependencies.  CI is specifically important for dynamical
languages and open-source code since it allows open-source projects to welcome
participation from many people while maintaining a high degree of reliability.

\paragraph{Websites to share knowledge.} Mutual aid and knowledge sharing
have been important aspects of the web since its inception and these practices
are now of foremost importance for many human activities.
%
In particular, learning programming and computer science is now greatly aided
by websites like the community driven encyclopedia
\href{https://www.wikipedia.org/}{Wikipedia} or the community driven forum
\href{https://stackoverflow.com/}{Stack Overflow}, where people write thousands
of questions and answers on programming each week.
% todo: remark from Julien Salort: it's old.
% av: So are technologies like git and hg, but we are putting together what
% characterizes our workflow, aren't we? We can emphasize that these forums
% often work only when there is a "thriving community" like the Python
% userbase.
IRC protocol has continued to sustain the test of time. Instant messaging
channels such as \codeinline{\#python} on Freenode facilitate interactive
discussions and a form of community-level support system --- a role which used
to be filled by mailing lists alone in the past. Alternatives such as Riot
(Matrix protocol), Gitter, and Slack have gained traction offering different
integrations solutions on top of instant messaging.
%
Through instant messaging a geographically separated team, as is the case with
FluidDyn project, can achieve near real-time collaboration without the hassle
of setting up video conferences or meetings.

\paragraph{Automatic web documentations built on servers.}  Documentation
generators such as Sphinx (for Python) and Doxygen (for many languages
including C, C++, Fortran and Python) ``automagically'' convert in-line
comments into documentations as PDF, HTML and more. Read the Docs has in the
recent years become the main host for building and publishing documentation on
servers as web pages, thus easing the process for developers and considerably
improving access to information.

% We need transitions and some words at the end of the subsections to help the
% readers to understand why we present what we present and what is the link
% with the FluidDyn project...

We have presented new methods and tools used in software development.  Taken
all together, they greatly improve our efficiency for collective work and open
possibilities of interactions that were unthinkable only few years ago.
%
For example, the success of community-driven software like \pack{astropy} (a
subject-area research library with \href{https://github.com/astropy/astropy}{more
than 240 contributors}) and \pack{scikit-learn} (a machine learning toolkit with
\href{https://github.com/scikit-learn/scikit-learn}{more than 1000 contributors})
would not have been possible without these new software development methods.
%
% pa: done: we need to mention in this paragraph existing community-driven
% subject-area research libraries (see the file
% ./communitydriven_research_soft.md).
%
As a scientists, we can wonder how to fully exploit these new possibilities to do
science.  The project FluidDyn is an attempt to help the fluid dynamics community
to advance in this direction.



\subsection{Python, a programming language adapted for open-science}

We now turn to the presentation of the
\href{https://www.python.org/}{programming language Python}, another
fundamental tool for the FluidDyn project.  Note that, this is not a detailed
technical presentation: we focus on the ideas necessary to understand why the
FluidDyn project is based on this language and on facts interesting for
potential FluidDyn users and developers.

Python community has thrived remarkably in the last few years, evident from the
fact that by the end of 2017, Python became the second most popular language in
terms of pull-requests in GitHub\footnote{See GitHub's annual report titled
\href{https://octoverse.github.com/}{ ``The State of the Octoverse 2017''}.} and
questions on StackOverflow\footnote{See
\href{https://stackoverflow.blog/2017/09/06/incredible-growth-python/}{ ``The
Incredible Growth of Python''}, a blog post by Stack Overflow.}.
%
TIOBE-index\footnote{\url{https://www.tiobe.com/tiobe-index}}, which assess the
popularity of a language based on number of skilled users, courses and
third-party vendors, ranks Python at number four. IEEE Spectrum\footnote{See
the analysis titled \href{https://spectrum.ieee.org/computing/software/%
the-2017-top-programming-languages}{``The 2017 Top Programming Languages''} on
IEEE spectrum.} ranks Python as the most popular language in 2017 with a metric
based on trends, demand by employers and users on open-source hubs.  All the
rankings do agree that Python usage has been steadily growing and is here to
stay.  Python largely outclasses languages like Fortran and Matlab according to
these statistics, and is now clearly one of the mainstream languages with C,
Java, C++ and Javascript.
%
Let us summarize some characteristics of the Python languages that have lead to
such incredible success.

% todo: DONE
% We need an introductory paragraph on how big is Python today
% + the fact that several scientists miss this fact.

% See the file ./python-one-main-languages.rst

The Python language has been designed to boost the communication of technical
ideas between humans. The resulting code is as simple as possible due to the
inherent emphasis on readability. Humans can focus on the ideas. Python differs
from other mainstream languages for a variety of reasons:

\begin{itemize}
\item \textbf{Aesthetics.} The syntax is nice and elegant. Blocks of code are
defined with the indentation.
\item \textbf{Code style.} There are explicit guidelines
(\href{https://www.python.org/dev/peps/pep-0008/}{PEP~8}) supporting the
regularity (and thereby readability) of the code and tools
(linters) to evaluate adherence to the style.
\item \textbf{Dynamic typing.} The variables (the ``names'' in the code) are
not attached for life to an object in memory\footnote{Opposite behavior
compared to static languages as Fortran, C and C++.}. The types of the objects
are inferred at run time from the code and the context, and in many situations,
one does not need to declare them explicitly.
%
% Being a strongly typed language\footnote{In contrast to weakly typed languages,
% such as JavaScript}, types of the objects are respected and implicit type
% conversions are rarely performed.
% av: "rarely / never"?
% eg: 'hello' * 2 -> 'hellohello'; 1/3 -> 0.3333333
%
% pa: I think we can not cover such complicate notion in this document... It is
% anyway not so important for our demonstration.
\item \textbf{Automatic memory management.}  The user need not handle the
memory allocation and deallocation by hand.  There are, of course, ways to
control the memory usage of a program, though not as precisely as for low-level
languages.
\item \textbf{Interpreted.} The standard way\footnote{There are also tools to
compile Python code (for example Cython and Pythran).} to run a Python code is
to ``interpret'' it.  A program called the interpreter executes the code
nearly\footnote{Actually, the execution of Python code is less simple and
involves an Abstract Syntax Tree.}  instruction-by-instruction, with very few
optimizations.
%
In contrast to other compiled languages such as Fortran, C or C++, there is no
proper compilation step: the code is not translated to optimized machine
instructions, but only to Python bytecode.
%
We will discuss shortly the disadvantages of this characteristic. Two evident
advantages are rapid development cycle and the possibility of interactive
workflow (usually with IPython or Jupyter notebooks).

Note that there are many implementations of Python interpreters\footnote{We can
cite CPython (written in C), Jython (Java), IronPython (C\#), PyPy (Rpython, a
subset of Python) and MicroPython/CircuitPython (C, targeted to
micro-controllers).}. The default and most widely used implementation is written
in C and is called CPython.

\item \textbf{Simple, yet powerful.}

\begin{itemize}
\item Python supports multiple programming paradigms, including imperative,
object-oriented and functional.

\item It is easy to interface with code written in other languages (in particular
C, C++ and Fortran).

\item Python comes with a large and high quality \href{https://docs.python.org%
/3/library/index.html}{standard library}.

\item Python can be deployed on many different machines with different operating
systems (Linux, Windows, macOS, Android) and architectures (from a microcontroller
--- with MicroPython or CircuitPython --- to a Blue Gene supercomputer).

\end{itemize}
\end{itemize}

Thus, Python is one of the best ``glue'' languages for fast prototyping.
Developing with Python is much faster and easier than with many other
languages.  Less bugs are introduced by the developers simply because there are
much less lines of code \cite{nanz_comparative_2015}.
%
Maintaining and modifying existing code is also eased by the very good code
readability associated with the language.

Another consequence of the apparent simplicity of Python is that the learning
curve has a nice shape. Learning Python is very easy at the beginning and the
complexity gradually increases as the level of the developer increases.
%
Therefore, Python is good for developers of all levels: very gentle for
beginners and very powerful for advanced users.

Python is an old language (first implementation in 1991!) but it continues to
evolve.
\begin{itemize}
\item A hard and controversial decision was taken to clean up Python from its
historical incoherences by having a new series of backwards-incompatible
versions of the language, Python 3.  Python 3.0 was released in December 2008.
After a long and difficult process, the transition from Python 2 to Python 3 is
now (2018) behind us for nearly all important packages\footnote{Close to
\href{https://python3wos.appspot.com/}{95\% of the most downloaded packages}
are Python 3 compatible now.}.  Especially for scientific applications, we can
now work only in Python 3. For instance, \href{https://github.com/%
numpy/numpy/blob/master/doc/neps/dropping-python2.7-proposal.rst}{future
releases of \Numpy} and of \href{http://www.python3statement.org/}{the main
scientific packages} will soon drop compatibility with Python 2. These changes
open doors to very clean, coherent and potentially faster Python
experience. Therefore, one should not use Python 2 for science anymore and
instead, adopt the newer versions ($>=3.5$ in 2018).

\item New versions of Python have nice new features to adapt to new usages and to
recent trends in computer science.  For example, three new features were
introduced in Python 3.5 (first released on September 2015): the \codeinline{@}
operator for matrix multiplication, the new \codeinline{async} and
\codeinline{await} keywords for concurrency and type hinting\footnote{See also our
paragraph on the issue of type checking in the following.} (with the module
\codeinline{typing} and an associated syntax presented in the
\href{https://www.python.org/dev/peps/pep-0484/}{PEP~484}).
\end{itemize}

\paragraphbf{An incredible success}

Being a very versatile language, Python is widely used for many different
applications:
\begin{itemize}
\item Simple scripting.

\item System, database and network administration.

\item GNU/Linux distribution software.

\item Python is now one of the main scripting languages to add programmability
to applications (for example Paraview, Visit, QGIS, Blender, ...).

\item Web servers.

\item Web scraping and data analysis.

\item Animation movies, game development and gaming.

\item Education. Python has recently become one of the most popular language
for teaching programming and computer science.

\item Science!
Python is now widely used in scientific applications. It has now a mature and
powerful scientific ecosystem with well-established based packages (\Numpy for
N-dimensional homogeneous arrays, \Scipy as the fundamental toolkit for
scientific computing, \pack{Matplotlib} for plotting and \pack{Pandas} for data
structures) and several more specialized packages (to name a few, \pack{h5py},
\pack{mpi4py}, \pack{skimage}, \pack{sklearn}, ...). There are now great (and
often shiny) tools for most of the applications. For example, the
\href{https://github.com/rougier/python-visualization-landscape}{%
landscape of visualization tools usable through Python} is impressive.
%
Python is one of the main languages for data science with packages such as
\pack{Pandas}, \pack{statmodels}, \pack{sklearn}, \pack{Keras}, \pack{TensorFlow}
and \pack{PyTorch}.

There are now ready-to-use ``batteries-included'' Python installers
(similar to Matlab), like in particular the
\href{https://en.wikipedia.org/wiki/Anaconda_(Python_distribution)}{%
freemium open-source distribution Anaconda}, so it is very easy to start using
Python for scientific purposes.

There are now nice Python integrated development environments (IDE) adapted for
scientists, for example \href{https://github.com/spyder-ide/spyder}{Spyder},
\href{http://www.pyzo.org/}{Pyzo} and
\href{https://jupyterlab.readthedocs.io}{Jupyterlab}.

\end{itemize}

Python has a very large and supportive community (see for example
the \href{http://stackoverflow.com/tags}{Stack Overflow tags}).
%
But Python is also supported by companies using Python and the open-source
dynamics.  Most of these companies that support Python are specialized in
web technologies and data-driven research.  But, recently, companies basing
their business
% av: earlier "computer and web technologies". Computer is quite obvious, isn't
% it?
on Python for science have started to become quite powerful in terms of
development. For example Anaconda Inc.\ is strongly implied in the development of
tools like \pack{Numba} and \pack{dask}. Hardware giants such as Intel and Nvidia
also contribute to the Python scientific ecosystem.


\paragraphbf{Some Python issues}

\paragraph{Since the standard Python interpreter does not do any proper
compilation,} Python code in some CPU bounded cases can be too slow.  Thus, one
has to use special tools to get optimized machine instructions to speedup
performance-critical code.

\textbf{Ahead Of Time compilation.}  Since Python is primarily a ``glue''
language, a common solution is to use code written in C, C++ or Fortran.
%
% todo: DONE
% add a reference to ctypes and cffi to call C with pure Python code
%
It is possible, without a compilation step, to call functions in C libraries
through pure python code using \pack{ctypes} standard module or the \pack{cffi}
package.
%
Libraries with Python bindings can also be used directly (for e.g., wrapped
using SWIG or Boost.Python) or one has to write extensions, i.e.\ Python
modules written in C or C++ using the CPython API that can be imported directly
from Python like any Python module.
%
However, to rely on such techniques to call C libraries would inhibit
development of extending functionalities, as they have a steeper learning
curve compared to Python.
%
Nevertheless, this strategy has given rise to the base modules of the scientific
Python stack, namely \Numpy, \Scipy and \pack{Matplotlib}.

Python developers do not actually have to write the compiled extensions in C,
C++ or Fortran, since there are tools to generate them automatically from
Python code (Pythran) or from Cython code. Cython \citep{behnel_cython2011} is a
smooth blend of two languages with possibility to add type declarations like in
C/C++, but with a syntax similar to Python.

Pythran \citep{guelton2015pythran, guelton2018pythran} is a quite recent Python
compiler which gives impressive results. It creates compiled extensions from pure
Python code with simple type annotations written as comments.
%
The resulting extensions are usually as fast as Fortran or C++ written by
non-specialists \cite[see, for example, benchmarks for the packages
\fluidpack{fft} and \fluidpack{sim},][]{fluidfft, fluidsim}.
%
They are created with a very interesting two-step compilations: first the code
is optimized at the Python level and then an automatically produced C++ code is
properly compiled.
%
This two-step compilation provides room for very clever optimizations as
demonstrated, for example, in
\href{http://serge-sans-paille.github.io/pythran-stories/%
being-more-than-a-translator.html}{%
this blog post by the main developer of Pythran}.
%
Pythran supports OpenMP pragma \citep{guelton2013compiling} and can use modern
vectorization with SIMD instructions \citep{guelton2014exploring}. Pythran
understands both Matlab-like vectorized code and C-like code with explicit loops
\citep{guelton2018pythran}.

We present in the two companion papers \citep{fluidfft, fluidsim} examples of
highly efficient Python codes, optimized using a combination of Cython and
Pythran extensions.

\textbf{Just In Time (JIT) compilation.} Another way to obtain optimized
machine instructions is by using JIT compilation, i.e.\ by compiling only the
critical code at run time.
%
This strategy can yield good results with other languages as for example Matlab
or Julia.
%
Pypy, an alternative interpreter written in Python, has a JIT compiler. However,
Pypy is not widely used for scientific applications mainly because of
compatibility problems with the extensions written for CPython.
%
However this could change since a recent version of Pypy (6.0) supports the main
packages of the Python scientific stack (\Numpy, \Scipy, \pack{Matplotlib},
\pack{Pandas}, etc.).

Adding a JIT to the interpreter CPython has been notoriously difficult (see
\href{https://faster-cpython.readthedocs.io/}{faster-cpython}). One of the
problem seems to be the C API provided in CPython and used by many extensions.
%
Python being first a glue language, a faster CPython with a JIT is actually not
so necessary for many purposes. Therefore, not as much money and work has been
put in accelerating Python than for example for Java and JavaScript.

Another strategy is to add a JIT to CPython through an external package
(\pack{Numba}) and to only compile the critical functions.  Moreover,
\pack{Numba} is particularly interesting because it can take advantage of the
GPU\footnote{See for example
\url{https://devblogs.nvidia.com/parallelforall/seven-things-numba/}}.


\paragraph{No type-checking.} In contrast to languages using Ahead Of Time
compilation, there is no type-checking in Python (and more generally in many
dynamical languages).  Type-checking is useful since it can avoid simple bugs.
However, it is not a good practice even with static languages to rely only on
type-checking to look for bugs in a code.
%
It is known that a rigorous set of unit tests is also needed to obtain more
reliable codes.
%
Dynamical languages rely a lot on unit tests so it is very important to put at
least a little bit of time and energy to write a decent battery of tests.
%
We have to mention the young and promising project
\href{http://mypy-lang.org/}{Mypy}\footnote{Guido van Rossum, the author and
``Benevolent Dictator For Life'' of Python, is involved in the project Mypy.},
which allows for Optional Static Typing and thus type-checking for
Python\footnote{We plan to add type hitting in the most important FluidDyn modules
to investigate how we can take advantage of type-checking with Mypy.}.

\paragraph{A lively, huge and thus complicated ecosystem.} As already
mentioned, Python is a versatile language. For many applications, the standard
library is insufficient and one needs external packages.
%
% For most applications there are several projects usable through Python.
Most of the time there are several Python packages for a certain application.
A user/developer has to make choices regarding the tools to use.  Especially
for the beginners who are not used to the open-source realm, it can be
difficult to understand a ``landscape of projects'' for a particular
application and to make good technological choices.

It is important to get a good introduction on how to work with the open-source
dynamics and the scientific Python ecosystem.  It can also be very useful to ask
to more advanced users and attend Python courses or workshops\footnote{This could
be regular academic courses, MOOCs or workshops like ones organized by
\href{https://software-carpentry.org/}{Software Carpentry}.}.
%
Of course, such abundance of projects can also be seen as an advantage.

\paragraph{Absence of multicore computational parallelism using threads.}
Threads behave like light subprograms. In principle, they can use the different
cores of the CPU at the same time but the CPython interpreter forbids that they
interpret python code at the same time\footnote{See
\href{https://opensource.com/article/17/4/grok-gil}{this blog article on ``How
to grok the GIL''} and
\href{https://faster-cpython.readthedocs.io/cpython37.html}{this technical
presentation by a CPython core developer}.}.
%
This limitation comes from an important detail in the implementation of the
CPython interpreter: a Global Interpreter Lock (GIL) to prevent race conditions
that could corrupt data.
%
It seems that the GIL greatly simplies the implementation of CPython and that it
is very difficult to remove it while keeping other nice technical properties of
CPython intact\footnote{See \url{https://wiki.python.org/moin/GlobalInterpreterLock}.}.
Note that Pypy also uses a GIL, while two other implementations of the Python
language, Jython and IronPython, do not have this limitation.

At any rate, threads using the interpreter do not use the CPU at the same time
so it is not possible to use threads to do multicore computational parallelism
with pure python code to accelerate CPU bounded tasks.
%
Note however, that threads can be (and are widely) used in Python for
concurrency, i.e.\ to perform I/O tasks (which does not need CPU) concurrently.
%
To do proper computational parallelism for CPU bounded tasks with the GIL, one has
to use other strategies: extensions (fine grain parallelism), multiprocessing
(coarse grain parallelism) and inter-process communication, for example, with MPI
(with mpi4py) or ZeroMQ.


\paragraphbf{Rich landscape of open-source languages useful for science}

So far, we have discussed in detail about Python. However, there are many
interesting open-source languages that can be used for scientific applications.
All have strengths and weaknesses.

Fortran remains widely used for computing codes, at least in fluid mechanics.
C++ is used in many recent developments. Its recent improvements (C++11, C++14)
make it a modern and very useful language.
%
JavaScript, Java, Scala, Smalltalk, Haskell, R, Julia, Perl and Lua are used for
some scientific codes. Go and Rust are quite young languages and are not (yet)
widely used for scientific applications.

It is of course very difficult to predict which technologies will be widely
used in even the near future (10 or 20 years for example).
%
From what we see, it seems that the idea of ``one language to do everything for
science'' will not succeed, at least soon.  This was maybe the project of Julia
creators, but despite the qualities of this language, it also has
limitations\footnote{For example C-like include instead of Python-like
import. From our point of view, a Pythonic flavor of Julia would be very
interesting.}.
%
It seems that a very important aspect in the future will be interoperability
between coexisting tools\footnote{See for example the cross-language development
platform \href{https://arrow.apache.org/}{Apache Arrow}.}.

So what about Python?  As already mentioned, it is today the language with the
strongest dynamics in science, data analysis and much more.  Several students
learn and will learn it. Many scientists in the future would, hopefully, be
able to appreciate these strengths and favour the Pythonic approaches.  With
such dynamics today, we can anticipate that the Python world will be able to
embrace new technological trends and to overcome some of its current
limitations.
%
It is reasonable to predict that Python will continue to shine as a language
(i) particularly good for human communication and to write ideas, (ii) for fast
prototyping and (iii) for interoperability with code written in other languages
(see for example the recent projects \href{https://cppyy.readthedocs.io}{cppyy}
and \href{https://github.com/pybind/pybind11/}{pybind11}, or
\href{https://github.com/PyO3}{the possibility to interact with Rust code}).

\paragraphbf{Summary: Python, a versatile Swiss Army Knife for the
scientists}

Python has become a great versatile tool for scientists.  The scientific
ecosystem of Python is nowadays really impressive.  The scientific Python
community is strong and quickly growing.
%
A scientist with good skills in Python can do most of what she/he needs to do in
programming (which does not mean that Python is the best language to do
anything!).
%
In contrast, no skill in Python is a real disadvantage for many tasks and for
employability.
%
Therefore, in most fields today, if a scientist has to acquire good skills in
one language, it is in Python.

Considering what we have presented on Python, we think that it is reasonable to
bet that it will become one of the \textit{lingua franca} in science and in
fluid dynamics.  This explains why we have chosen it to be the main language
for the project FluidDyn.


\subsection{Productivity at individual, group and community levels}

% remark Julien Salort "It is well known that": bad
% pa: common sense says ?
% It is well known 
It is not surprising
% av: "It is not surprising"
%
that short-term efficiency and long-term efficiency are
sometimes incompatible.
%
Quick and dirty scripts can be efficient in the short term, but such a workflow
has a negative impact on a longer time scale.

Similarly, we can also differentiate productivity at different scales of
collaboration.
%
An individual can be very efficient with a particular tool, but the same tool
can be very inefficient to collaborate even with co-workers.
%
A group can be very efficient with a code which is closed-source.  If this
strategy can be efficient for the group (at least in the short term), it leads
to waste for the community. Other groups will need to develop codes with the
same features.  Good ideas will be spread in different codes and since the
number of users and developers are smaller for each of these codes, the code
quality will not improve as fast as it would with an open-source strategy. In a
nutshell, as this proverb says, \textit{``If you want to go fast, go alone. But
if you want to go far, go together.''}

Thus, a community using open-source methods can be efficient to co-develop its
tools. However, some of its members have to spend more energy to bear this
dynamics, which implies that they somehow sacrifice their individual efficiency.
We see that short-term efficiency, long-term efficiency and productivity at
different levels can often be incompatible and that we have to consider these
conflicts when choosing between different technologies.

\subsection{Programming in the field of fluid mechanics}

We study fluid mechanics via laboratory experiments, in situ measurements,
analytic computation, numerical simulations and data processing.
%
Nowadays, programming is involved in all methods, and thus it is difficult to do
fluid mechanics without software development.
%
However, the level in software engineering is on average very low in the
community.
%
Even today, it is quite commonplace to start a PhD without any serious training in
GNU/Linux and modern programming tools.
%
A large majority of scientists and technicians are also unaware of the new
challenges and opportunities of open-source.

Since fluid dynamics is an engineering science, closed-source commercial software
have historically been dominant in the field.
%
Few people are aware of the vicious circle of the closed-source model for people
and groups relying on it.
%
This mechanism is nevertheless very damaging. In the closed-source model, a group
pays for a license or a new development. The group does not learn how to develop
what has been paid for. The company, on contrary, stands to gain from profits and
the feedback generated from users, through which it improves the product sold.
The group produces codes, books, courses using the closed-source product or
acquire knowledge on how to use the closed-source product, so that the group is
increasingly dependent on the product and is ready to pay more for it.
%
Such vicious circles are difficult to break, but not impossible. This can be
done by introducing alternative open-source solutions, bringing in positivity to
this circle.

We have seen that it was very common to rely on commercial companies and
closed-source software for most of the new developments in fluid mechanics.
%
Nevertheless there are also a lot of codes produced by scientists in the
field. But without the technical knowledge on how to work collectively on code,
this production is often of bad quality (compared to the information technology
industry standard) and it is basically doomed to be abandoned after its use.
%
However, it is natural to try to reuse code or at least to build the next steps
on what has been done before.  Thousands of hours of highly qualified people
are spent in trying to understand and reuse codes badly written with
inappropriate languages!

On a technical point of view, the majority of the coding in the field involve a
mix of Fortran/C or C++, shell languages (as Bash) and Matlab. For experiments,
the graphical programming environment
\href{http://www.ni.com/en-us/shop/labview.html}{Labview} is dominant for
control of physical objects and acquisition and Matlab is used for data
processing.
%
Often, languages are used for things for which they are not adapted.
%
Fortran, C or C++ have not been designed for fast prototyping of complex
programs.
%
It is known that for scientific purposes, one should restrict shell scripting
to extremely simple tasks.
%
Similarly, using Matlab for developing complex programs is technically a
questionable choice.
%
These technical solutions also represent a problem for code reuse, sharing and
collaborative development.
%
We are now going to present factual arguments explaining why we think that the
massive usage of Matlab decreases the collective efficiency of the community.

\paragraphbf{Discussion on Matlab, Labview and Mathematica}

% see http://www.pyzo.org/python_vs_matlab.html
% see https://www.linkedin.com/pulse/matlab-vs-python-jan-rhebergen

Matlab is a closed-source proprietary numerical computing environment.
%
It is a good tool for simple processing with matrices, image processing, data
visualization and certain niche applications. The language is well adapted for
these tasks. The development environment is nice and the interpreter is quite
fast, especially now that it has a JIT compiler.

However, the language suffers from serious technical issues which make Matlab
an inadequate tool
% \citationneeded[opinion] % pa: I think what we say is fine for an article. We
% are not writing a wikipedia page which has the ambition to be neutral. Here,
% we gives an argued opinion, which is totally normal in a scientific article.
% av: fair enough :)
for doing more than simple processing and data plotting. The comparison with
Python often debated upon. We list a few striking Matlab weaknesses:

\begin{itemize}
\item One file for each function (or class) available outside the file where it
is implemented. No notion of package or module.

\item No real organization of the standard library. All built-in and
user-defined functions are available in a huge flat namespace. No import
mechanism and as a result we cannot deduce, simply by looking at the code,
where a function comes from.

\item A standard way to organize multi-file code is to write scripts that
modify and define global variables (see for example
\href{https://www.damtp.cam.ac.uk/user/jrt51/files/diablo\_mat.tar.gz}{%
this Matlab version of the code Diablo}).  It is so simple to do this that we
can assert the language strongly encourages this practice.  Matlab files are
not self consistent, i.e.\ it is normal to use in a file a global variable
defined outside of the file.

\item Very bad default argument mechanism.

Default arguments for a function is a very common feature is many programming
languages. In Python, we can write:
\begin{minted}[fontsize=\footnotesize]{python}
def myfunc(a, b, c=1, has_to_print=True):
    if has_to_print:
        print('a =', a, 'b =', b, 'c =', c)
    return c * (a + b)
\end{minted}

An implementation for approximately the same behavior in Matlab could be (as
advised in the official Matlab documentation):
\begin{minted}[fontsize=\footnotesize]{matlab}
function ret = myfunc(a, b, varargin)
    if nargin < 2 | nargin > 4
        error(['The number of arguments has to be ' ...
               'greater than 2 and lower than 5'])
    end
    if nargin == 4
        has_to_print = varargin{2};
    else
        has_to_print = 1;
    end
    if nargin >= 3
        c = varargin{1};
    else
        c = 1;
    end

    if has_to_print
        disp(['a = ' num2str(a) '; b = ' num2str(b) '; c = ' num2str(c)])
    end
    ret = c * (a + b);
end
\end{minted}
Even for a very simple function, there are many places where bugs can be
introduced and the code is much less readable than the corresponding Python
code.

\item Less intuitive syntax for string operations in Matlab, compared to
Python. Take for example for string comparison, \codeinline{strcmp(s1, s2)} in
Matlab versus \codeinline{s1 == s2} in Python or for look up,
\codeinline{contains(s1, pattern)} in Matlab versus \codeinline{pattern in s1}
in Python.
% \item Less intuitive and clear syntax for exception
% handling. \citationneeded[opinionated].
% pa: Matlab has now a MException class. The syntax is not nice, but it is
% not so important.
\item Less clear and intuitive model and syntax for object oriented
programming, compared to Python. As for functions, one file per user-defined
class is necessary.
% \citationneeded[opinionated].
\item Parenthesis used for both function calls and indexing, causing ambiguity.
\item Matlab codes are usually full of \codeinline{;}, \codeinline{.*},
\codeinline{./}, \codeinline{\&\&} and \codeinline{||}, which make them quite
``noisy''.  Moreover, there is nothing like the Python PEP~8. This makes a
large proportion of the Matlab codes difficult to read and understand, due to
lack of a consistent code style.
\item The syntax \codeinline{a(100, 100) = 1;} to create and extend matrix is a
very good way to hide bugs. The code \codeinline{a = eye(2); a(i0, i1) = 1;}
will never raise any error regardless the values of \codeinline{i0} and
\codeinline{i1}! No error is raised if a user misspell the variable
\codeinline{a} and write something like \codeinline{aa(100, 100) = 1;}.
\end{itemize}

Now we turn our attention, from structural issues of Matlab syntax, to more
practical constraints its use brings about.  By definition of a closed-source
software, Matlab is a big black box: it is impossible to study the
implementation of its functions.

Matlab is not free. The price for one license for non-commercial use is not
very expensive for most research and teaching institutes. However, when we
start to add up the price of the toolboxes necessary to run most codes, it
starts to become substantially large for some institutions.  The price of
Matlab quickly escalates when deployed on a cluster with MPI, since a licence
is required per user, per node!
% av: not per process.
% see: https://www.nas.nasa.gov/hecc/support/kb/running-matlab-applications-in-parallel_522.html

It would be interesting to estimate what could be done with open-source
solutions with a part of the amount paid by public research institutes. How
many developers could be paid to improve the alternative open-source solutions?
%
This question raises the issue of funding scientific open-source software.

Moreover, it is well known that the lack of license for the personal computers
of staff is sometimes solved with cracked versions of Matlab. How can serious
research projects rely on such illegal methods?
%
Note that open-source Matlab interpreters do exist, such as Octave and Scilab.
%
% pa: warning! Octave has a JIT compiler!
%
% Unfortunately, these are not serious alternative today. Their user
% community is too small and they are not very efficient, especially now that
% Matlab uses a JIT compiler and that optimized Matlab codes rely on it.

The recent open-source and Python dynamics represent a hard punch for Matlab,
which is losing the dominant position it has in some domains.
%
Note that this is a very recent development. The Python scientific ecosystem can
now compete in nearly all domains, but this was not the case typically 5 years
ago.
%
There are still a large number of researchers and teachers in sciences who are
strongly attached to Matlab mainly because they have used it a lot.
%
It is, of course, difficult to change habits especially when those are efficient
(for a short time-span and on an individual or group scale) enough for what you
do.

It is normal that Mathworks, the company which makes money with Matlab, spreads
\href{https://www.mathworks.com/products/matlab/matlab-vs-python.html}{arguments
against Python and its scientific ecosystem}.  Some of them are valid, for
example the issue of the lively, huge and thus complicated scientific ecosystem
that we have already discussed.  Other arguments are unfair or no longer valid
nowadays.
% Here, we remove a long argument with Mathworks (see the file
% old_arguments_mathworks.tex)
% av: Less provocative, that is good. But we can still add a footnote linking
% the Pyzo page / LinkedIn page comparing python and matlab.
%
% We replace by this:
The scientific Python ecosystem is now very mature, robust, well documented and
fully integrated.  Many of this components are supported by professional
developers paid by companies or research institutes, for example \pack{Numba}
(Anaconda, Nvidia), \pack{TensorFlow} (Google), \pack{PyTorch} (Facebook, Twitter,
Nvidia), Scikit-learn (INRIA), Mercurial (Facebook).
%
The support by the Python community is both very reactive and of very good
level (for example in Stack Overflow, and issues page on major repositories)
and it is now possible to pay companies to get professional support on
scientific Python (for example
\href{https://software.intel.com/en-us/distribution-for-python/get-help}{Intel}
and \href{https://www.anaconda.com/enterprise/}{Anaconda}).
%
Finally, with the tools available today, with little effort we can get a very
efficient program using only Python code.

To conclude on Matlab, some people may use it efficiently for their work, but it
starts to be a problem for the collective efficiency of the community. Except
from the existence of a large code base difficult to translate in better
languages or for some very specific applications, it seems difficult today to
find a good technical reason to continue using Matlab.

The same remarks can be made for Labview. It is a graphical programming
language and the programs are saved as binary files. This is a problem because
first it is impossible to read a Labview program without Labview and, more
importantly, it forbids the use of source management tools which are so
important for collective work and open-source.

Matlab and Labview also lead to problems for students, who learn bad coding
habits and languages that are much less in demand by employers than for example
Python.

For symbolic computation, some specialists use Mathematica, which is a good and
very expensive proprietary closed-source mathematical symbolic computation
program.
%
Sage and Sympy are two complementary open-source alternative both based on
Python.


\paragraphbf{Different models for software development in fluid mechanics}

Proprietary codes tend to dominate the field of fluid mechanics. This is true
for industrial Computational Fluid Dynamics (CFD), for example with the Ansys
suite and also for acquisition and analysis of images of fluid with the
two companies Dantec and Lavision.
%
Even certain softwares developed by researchers in academic laboratories
also follow a closed-source model, for example
\href{http://www.damtp.cam.ac.uk/user/fdl/digiflow/index.htm}{Digiflow}.
%
Closed-source or even undistributed software is a widespread model in the
field.
%
The arguments against sharing are diverse:

% todo: DONE
% pa: I am really uncomfortable with this presentation which gives the
% impression that we think that the arguments against sharing are always wrong.
%
% We have to make it clear that it is not the case and that we do respect how
% other people think and work.
% av: I have commented out specific counter arguments, instead added two
% sentences after the bullets
\begin{itemize}
\item \textit{``We do not share to keep a comparative advantage.''}

%	Sharing code leads to reproducibility, making related
%	publications more attractive to readers. The authors gain
%	comparative advantage by earning citations.

\item \textit{``We do not share because we do not provide support.''}

%	Support and documentation is invaluable, of course. But sometimes, a code
%	with comments can be sufficient, which is a common practice.

\item \textit{``We do not share because people would not be able to
correctly use the code or interpret the results.''}

% pa: this one does not respond to the problem... (bad use of the program)

%	Poorly written code is infinitely more useful than no code at all.

\item \textit{``We do not share because we want to control industrial usage.''}

%	Proper choice of license can legally restrict and permit industrial
%	usage at the discretion of the developer.

\item \textit{``We do not share because we do not want people to review
and criticize our code.''}

%	People are, in general, kind and welcoming in the open-source community.

\item \textit{``We do not share, because in this way, people will think our
project has more value.''}

% pa: this one looks too "religious" to me :-) but ok...
%	True value is gained by furthering science and dissemination of ideas, not by
%	fabricating an appearance.
\end{itemize}

The flip-side involves releasing a well-written, documented, easy to use and
maintain source code with appropriate license to ensure credit where it is due.
This can potentially add value for the community through added transparency in
publications, and in return, value for the authors of software in terms of
recognition and citations. We have only looked at the most ideal scenarios and
the question of soundness of the above arguments is left as an open question to
the reader.

Between closed-source and open-source, is a grey area gathering different
practices.
%
It is quite common to share only for ``friends'' to control the dissemination
of the software.
%
This model can be institutionalized with an official group, for example with
the code \href{https://www.coria-cfd.fr/index.php/YALES2}{Yales2}.
%
% av: Institutionalized? I don't understand.
With such a model, the lack of a proper open-source license can naturally lead
to problems and is a limitation for users/developers.
%
Another common practice is to share without repository (for example
\href{http://www.damtp.cam.ac.uk/user/jrt51/files.html}{Diablo} or
\href{http://yakari.polytechnique.fr/people/deloncle/ns3d.html}{NS3D}) and/or
without license (\href{https://choosealicense.com/no-permission/}{meaning that the
code is actually under exclusive copyright}).

Finally, several scientists in the fluid dynamics community have used
open-source methods in their work for a long time.
%
There are a lot of open-source research codes written with proprietary tools
(typically written in Matlab, see for example
\href{http://pivlab.blogspot.de/}{PIVlab},
\href{http://servforge.legi.grenoble-inp.fr/projects/soft-uvmat/}{UVmat} and
\href{https://de.mathworks.com/matlabcentral/fileexchange/10902-pivmat-4-00}{PIVmat}).
%
Let us first say that these codes are a great wealth for the community at least
for three reasons: (i) they can be used as they are, (ii) people using these
codes can be more open to the open-source dynamics, and (iii) open-source
research projects using open-source tools can exploit the ideas in these codes.
%
Let us also point out a drawback of such a model. We do not pay the authors, but
people do need to pay to use the software. Even the authors of the software give
money to the same company to be able to use their software. Do they realize that
when they work on these programs, they work (often, for free) for the company that
sells the proprietary tool?

Some proper open-source codes have emerged, for example
\href{https://nek5000.mcs.anl.gov/}{NEK5000} (Fortran),
\href{https://www.openfoam.com/}{OpenFOAM} (C++),
\href{http://basilisk.fr}{Basilisk} (C) and
\href{http://channelflow.org/}{Channelflow} (C++) and
\href{https://www.code-saturne.org}{Code\_Saturn} (C/Fortran).
%
Big companies have started to use open-source development for fluid mechanics
application. For example, Volkswagen group (Volkswagen, Audi, Seat, Porsche,
Skoda, ...) has used OpenFOAM since
2006\footnote{\url{https://www.nas.nasa.gov/assets/pdf/ams/%
2014/AMS_20141202_Othmer.pdf}}.
%
EDF, the main French electric utility company has also a interesting
\href{http://linuxfr.org/news/strat\%C3\%A9gie-open-source-\%C3\%A0-edf-rd}{%
open-source strategy}.  It has produced many programs and in particular the CFD
solver \href{https://www.code-saturne.org}{Code\_Saturn}.
%
EDF is also involved in the development of
\href{https://www.paraview.org/}{Paraview}\footnote{\url{https://blog.kitware.com/introducing-some-new-paraview-5-0-features/}}.

Finally, scientists have started to use Python to study fluid mechanics,
especially through CFD codes (\href{http://dedalus-project.org/}{Dedalus},
\href{https://github.com/spectralDNS}{SpectralDNS},
\href{https://github.com/mikaem/Oasis}{Oasis}, \href{http://pyfr.org/}{PyFR}
and \href{https://fenicsproject.org/}{FEniCS}) or through data analysis
(\href{http://www.openptv.net/}{OpenPTV},
\href{https://github.com/jr7/pypiv}{PyPIV}).
%
The packages of the FluidDyn project are also part of this trend.

\section{FluidDyn project: Breaking the status quo}
% \section{Implementation and architecture}

We have seen that there is a strong dynamics in play around the use of
computers (in particular with the web) and that this creates very efficient
tools and methods for collective work and software development.
%
The potential of these tools is not fully exploited yet, in the field of fluid
mechanics.

FluidDyn is a project to foster open-science and open-source coding in Python
in the field of fluid mechanics.
%
The project envisages to provide the technical framework to allow collaborative
development of tools useful for the fluid mechanics community, and to do better
science with open methods.
%
% It first started with one package.  Soon we had to prefer a decentralized
% organization with one common package and other specialized packages.
% av: mentioned in the next subsection
We provide examples of solutions for:
\begin{itemize}
\item Good coding practice with readable and easy to comprehend Python code
(PEP~8).
\item Source control management (Mercurial) and forge (Bitbucket) simple for
the new comers.
\item Packaging and installation procedure. All packages are available from
\href{https://pypi.org/}{Pypi} and can be install simply with
\href{https://pypi.org/project/pip/}{pip}, the standard Python installation
tool.
\item Licenses: depending on the packages, we choose to use the CeCILL-B or the
CeCILL licenses. These licenses are a BSD compatible and a GPL compatible
licenses adapted to both international and French legal
matters\footnote{\url{http://www.cecill.info/licences.en.html}}.
\item Documentation produced with standard and up-to-date tools: Sphinx,
Anaconda and Jupyter. Built and hosted online at
\href{https://readthedocs.org/}{Read the Docs}. Can also be generated offline.
\item Unittest and continuous integration with Bitbucket Pipelines and Travis.
\end{itemize}
We hope that such clean framework will allow contributions from scientists in
the field and that we can build together a nice, user-friendly and efficient
ecosystem specialized in research and teaching in fluid dynamics.


% \textcolor{blue}{How the software was implemented, with details of the
% architecture where relevant. Use of relevant diagrams is appropriate. Please
% also describe any variants and associated implementation differences.}

\subsection{Organization of the code in packages}

FluidDyn was originally intended to be a single package to perform experiments
and simulations. Since a typical user may not be involved in both experiments
and simulations and also, due to increasing complexity as a virtue of rapid
development cycle made the need to decentralize FluidDyn evident.
Now, FluidDyn project hosts a number of specialized packages, namely:

\begin{itemize}
\item \fluidpack{dyn}: The base package which contains pure-python code that can
be reused in scripts or in specialized FluidDyn packages. It also contains codes
for miscellaneous command-line utilities useful for a typical fluid dynamics user.

The code of this package is presented in further detail in its documentation
(\url{https://fluiddyn.readthedocs.io}) and some prominent features are presented
in the following subsection.

\item \fluidpack{fft}~\cite[see the companion paper][]{fluidfft}: a package which
provides C++ and Python classes unifying various libaries to perform Fast Fourier
Transform (FFT) in sequential and in parallel.

\item \fluidpack{sim}~\cite[see the companion paper][]{fluidsim}: Numerically
oriented framework to run sequential and parallel Computational Fluid Dynamics
(CFD) simulations and on-the-fly post-processing for a variety of problems
(Navier-Stokes, Shallow Water, F\"oppl von K\'arm\'an equations, to name a few).
A study using \fluidpack{sim} has just been published in Physics of Fluids
\cite[]{LindborgMohanan2017}.

\item \fluidpack{lab}: Package to handle laboratory experiments. Primarily used to
communicate with various hardware devices such as motors and pumps, to handle I/O
between sensors, and to store data.
%
Experiments using \fluidpack{lab} have been carried out in the DAMTP fluid
laboratory \cite[Cambridge, UK. cf.][]{LeclercqPartridgeAugierDalzielKerswell2016}, in ENS Lyon laboratory
\cite[Lyon, France][]{salort2018} and at LEGI \cite[Grenoble, France][]{ISSF2016}.

\item \fluidpack{image}: Scalable image processing package which implements
various algorithms to calibrate cameras, to preprocess images, to do Particle
Image Velocimetry (PIV) and to postprocess data.
%
Fluidimage is used to process images taken during experiments performed in the
Coriolis platform at LEGI \cite[]{ISSF2016}.

\item \fluidpack{foam}: Small package to load OpenFoam data and plot them.

\item \fluidpack{coriolis}: Small package used to carry out experiments in the
\href{http://www.legi.grenoble-inp.fr/web/spip.php?article757}{Coriolis platform}
(a large rotating platform participating in the European consortiums Euhit and
Hydralab) and open the data obtained \cite[see, for example,][]{ISSF2016}. One of
the motivations behind creating this package is to study how to use open-source to
create and share open-data.

\end{itemize}

A detailed presentation on the above packages can be found in their respective
documentations on the web and for \fluidpack{fft} and \fluidpack{sim} in the
two companion papers \cite[]{fluidfft, fluidsim}.  This article will now focus
on the base package \fluiddyn.

All functions and classes defined in \fluiddyn are pure Python elements, meaning
that no extensions are implemented in \fluiddyn.  Thus the package \fluiddyn is
extremely easy to install with just a \codeinline{pip install} command and no
compilation.

The code base was designed to follow Python 2.7 syntax during its genesis. Now,
it has been made forward compatible with Python 3 through the use of external
package future.

\subsection{API of the Python library \fluiddyn}

The package \fluiddyn is organized into five sub-packages: 
\begin{itemize}
\item \codeinline{fluiddyn.io}: This subpackage provides utilities for
input/output to different file formats.
\item \codeinline{fluiddyn.util}: Miscellaneous utilities.
\item \codeinline{fluiddyn.calcul}: Modules for numerical computations.
\item \codeinline{fluiddyn.clusters}: Classes to launch jobs on HPC clusters.
\item \codeinline{fluiddyn.output}: Utilities to produce scientific outputs
(figures and videos).
\end{itemize}

Sub-packages \codeinline{io}, \codeinline{util} and \codeinline{calcul} are the
largest in terms of lines of code and provides the Application Programming
Interfaces (API) to support \fluidpack{sim}, \fluidpack{lab} and
\fluidpack{image}.

\section{Conclusion}

% FluidDyn is an attempt to set off collaborative dynamics based on open-source
% development in fluid dynamics research.
% %
% We shall try, with this project, to explore the possibilities of open-source in
% science and fluid dynamics, by fully exploiting the new open-source tools and
% methods presented in the first subsections of the introduction.
%
% % todo: again mention some success of community-driven subject-area research
% % libraries? See the file ./communitydriven_research_soft.md
%
% The project is right now in a very preliminary stage.  Packages are actively
% evolving with interesting features and a framework for collaborative
% development, packaging, documenting and testing is now well set.
% %
% However, the community around the project is currently tiny and now, we have to
% work on attracting users and developers since an active community is a
% criteria for success and sustenance of an open-source project.
%
% Will people use the FluidDyn tools and collaborate through the project
% FluidDyn?
% %
% There are clearly many challenges and potential barriers:
% \begin{itemize}
% \item Some habits in the community.
% \item Knowledge and skills in the community. For example, only few people use
% issue tracker and pull requests.
% % from Julien Salort: bad!
% % \item Misunderstanding between colleagues and in particular colleagues with
% % responsibilities and power.
% \item Lack of a business model for open-source software in science.  A good
% quality open-source software has a cost that institutions should be willing
% to fund and support.
% \item Lack of recognition of the work spent in open-source, in particular for
% precarious scientists.
% \end{itemize}
%
% On our side, we also have positive points.
% %
% The quality of the tools we use (Python and its scientific ecosystem,
% Mercurial, Read the Docs, Jupyter, ...) is impressive.
% %
% Scientific code is done to be read and to transmit ideas. To this effect, Python
% is among the best languages today.
% %
% We can benefit from the dynamics of Python and of emerging subjects like deep
% learning and the Internet of Things.
% %
% Finally, if we manage to gather a community of users and of developers, the
% collective efficiency related to open-source methods and tools can be a strong
% booster.
