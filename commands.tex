% Insert HERE any additional commands


% as examples here are some commands from the JFM template:
%
\newcommand\etal{\mbox{\textit{et al.}}}
\newcommand\etc{etc.\ }
\newcommand\eg{e.g.\ }

\providecommand\bnabla{\boldsymbol{\nabla}}
\providecommand\bcdot{\boldsymbol{\cdot}}

\newcommand\Rey{\mbox{\textit{Re}}}  % Reynolds number
\newcommand\Pran{\mbox{\textit{Pr}}} % Prandtl number, cf TeX's \Pr product
\newcommand\Pen{\mbox{\textit{Pe}}}  % Peclet number


% Commands used for chapter open science
\newcommand{\fluidpack}[1]{\href{https://fluid#1.readthedocs.io}{%
\codeinline{fluid#1}}}

% \newcommand{\codeinline}[1]{\mintinline{python}{#1}}
\newcommand{\codeinline}[1]{\texttt{#1}}

\newcommand{\fluiddyn}{\fluidpack{dyn}\xspace}

\newcommand{\Numpy}{\codeinline{Numpy}\xspace}
\newcommand{\Scipy}{\codeinline{Scipy}\xspace}

\newcommand{\pack}[1]{\codeinline{#1}\xspace}

\newcommand{\mako}{\href{http://www.makotemplates.org/}{\pack{mako}}}

\newcommand{\libpack}[2][]{%
  \ifstrequal{#2}{FFTW}{%
    \href{http://fftw.org}{#2}}{%
  \ifstrequal{#2}{MKL}{%
    \href{https://software.intel.com/en-us/mkl}{#2}}{%
  \ifstrequal{#2}{PFFT}{%
    \href{https://www-user.tu-chemnitz.de/~potts/workgroup/pippig/software.php.en}{#2}}{%
  \ifstrequal{#2}{P3DFFT}{%
    \href{http://p3dfft.net}{#2}}{%
  \ifstrequal{#2}{2decomp\&FFT}{%
    \href{http://www.2decomp.org}{#2}}{%
  \ifstrequal{#2}{cuFFT}{%
    \href{https://docs.nvidia.com/cuda/cufft/index.html}{#2}}{%
  \ifstrequal{#2}{clFFT}{%
    \href{https://clmathlibraries.github.io/clFFT/}{#2}}{%
  \ifstrequal{#2}{FFTPACK}{%
    \href{http://www.netlib.org/fftpack}{#2}}{%
  \ifstrempty{#1}{%
    #2
  }{%
    \href{#1}{#2}}
  }}}}}}}}\xspace  % Close the if-else-if tree above!
}


% other
\newcommand{\p}{\partial}
\newcommand{\Dt}[1]{\ensuremath{\frac{D #1}{Dt}}}

% Wikipedia-style "citation needed" macro
\newcommand{\citationneeded}[1][]{\textsuperscript{\color{blue} [citation needed: #1]}}

\newcommand{\paragraphbf}[1]{\subparagraph{\bf #1}}

\renewcommand{\descriptionlabel}[1]{\hspace{\labelsep}\textit{#1}}

% Extend href with footnote url
\newcommand{\fnref}[2]{\href{#1}{#2}\footnote{\url{#1}}}

% !Undefined control sequence.
% https://tex.stackexchange.com/questions/8361/latex-ieeetran-cls-use-titlesec-package
% \newcommand{\tocsubparagraph}{}

% From shallow water paper
\newcommand\shalf{\ensuremath{{\scriptstyle\frac{1}{2}}}}
\newcommand\sh{\ensuremath{^{\shalf}}}
\newcommand\smh{\ensuremath{^{-\shalf}}}
\newcommand\squart{\ensuremath{{\textstyle\frac{1}{4}}}}
\newcommand\thalf{\ensuremath{{\textstyle\frac{1}{2}}}}
\newcommand\Gat{\ensuremath{\widetilde{G_a}}}
\newcommand\ttz{\ensuremath{\rightarrow 0}}
\newcommand\ndq{\ensuremath{\frac{\mbox{$\partial$}}{\mbox{$\partial$} n_q}}}
\newcommand\sumjm{\ensuremath{\sum_{j=1}^{M}}}
\newcommand\pvi{\ensuremath{\int_0^{\infty}%
  \mskip \ifCUPmtlplainloaded -30mu\else -33mu\fi -\quad}}


\newcommand{\uu}{\textbf{u}}
\newcommand{\xx}{\textbf{x}}
\newcommand{\kk}{\textbf{k}}
\newcommand{\JJ}{\textbf{J}}
\newcommand{\rr}{\textbf{r}}
\newcommand{\ff}{\textbf{f}}
\newcommand{\FF}{\textbf{F}}
\newcommand{\baa}{\textbf{a}}
\newcommand{\bb}{\textbf{b}}
\newcommand{\NN}{\textbf{N}}





\newcommand{\ErtelPV}{\mathcal{Q}}

\newcommand{\fdiss}{f_{\mbox{\scriptsize diss}}}

\newcommand{\D}{\mbox{D}}

\newcommand{\eez}{\boldsymbol{e_z}}
\newcommand{\scalarprod}[2]{\big( #1 \, , \ #2 \big)_{\kk}}

\newcommand{\mean}[1]{\langle #1 \rangle}

\newcommand{\meane}[1]{\langle #1 \rangle}
% \newcommand{\meane}[1]{{\langle #1 \rangle_{\hspace{-0.4mm}e}}}

\newcommand{\meanx}[1]{{\langle #1 \rangle_{\hspace{-0.4mm}\mbox{\scriptsize$\xx$}}}}

\newcommand{\meant}[1]{{\langle #1 \rangle_{\hspace{-0.4mm}\theta}}}

\newcommand{\means}[1]{{\langle #1 \rangle_{\hspace{-0.4mm}s}}}

\newcommand{\shocks}{ {\mbox{\tiny shocks}} }

\newcommand{\kmax}{k_{\mbox{\scriptsize max}}}
\newcommand{\kdiss}{k_{\mbox{\scriptsize diss}}}


\newcommand{\mA}{\mathcal{A}}


\newlength{\halfwidth}
\setlength{\halfwidth}{2.6in}

\newcommand{\PA}[1]{{\color{green}#1}}

\newcommand{\Add}[1]{{\color{blue}#1}}
% \newcommand{\Add}[1]{#1}
\newcommand{\Remove}[1]{{\color{red}\st{#1}}}
% \newcommand{\Remove}[1]{}

\newtheorem{lemma}{Lemma}
\newtheorem{corollary}{Corollary}

% Personal
\newcommand{\mfivethird}{\ensuremath{\frac{-5}{3}}}
