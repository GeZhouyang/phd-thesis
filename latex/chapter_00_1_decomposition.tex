\section{Shallow water equations}

\textbf{The text below needs to be given a proper structure!}

\subsection{Governing Equations}

The governing equations for a shallow layer of fluid are:
\begin{align}
    \label{dtu0} \partial_t \mathbf u & = - (\mathbf{u}.\nabla) \mathbf{u} - c^2 \nabla h - f\mathbf{e_z} \times \mathbf u \\
    \label{dth} \partial_t h         & = - \nabla. (h \mathbf u)
\end{align}
Equation \eqref{dtu0} may also be written in the \textit{rotational form} as:
\begin{equation}
    \label{dtu}
    \partial_t \mathbf u
    = - \nabla |\mathbf u|^2/2 - c^2 \nabla h - \zeta \times \mathbf u
\end{equation}
where, \gls{zeta}
represents absolute vorticity, i.e. the sum of vorticity and system rotation.
Furthermore, equations \eqref{dtu0} and \eqref{dth} can be combined to form an
equation for the \textit{total mass flux}, \gls{J} $= h\mathbf{u}$:
\begin{equation}
    \label{dtJ}
    \partial_t \mathbf J = -(\mathbf{u}.\nabla)\mathbf{J} - \nabla(c^2h^2)/2 -
    \zeta \times \mathbf J - (\nabla. \mathbf J)\mathbf u
\end{equation}
Similarly, an equation for the \textit{displaced mass flux}, \gls{M} $ =
    \eta\mathbf{u}$ is:
\begin{equation}
    \label{dtM}
    \partial_t \mathbf M = -(\mathbf{u}.\nabla)\mathbf{M} - \nabla(c^2\eta^2)/2 -
    \zeta \times \mathbf M - (\nabla.\mathbf{u} + \nabla.\mathbf{M})\mathbf u
\end{equation}
For a divergence free flow, a Poisson equation for \emph{h} can be formulated.
Taking divergence of \eqref{dtu},
yields the Poisson equation:
\begin{equation}
    \label{poisson}
    \nabla^2 h = \frac{1}{c^2} \left[ \nabla.(\zeta \times \mathbf u )
        - \nabla^2 \frac{|u|^2}{2} \right]
\end{equation}
Applying fourier transform, the spectral counterpart for \eqref{poisson} in
tensor notation is:
\begin{equation}
    \label{poisson_fft}
    -\kappa^2 \hat{h} = \frac{1}{c^2} \left[ ik_i (\widehat{\epsilon_{ijk} \zeta_j
            u_k})
        + \kappa^2 \frac{\widehat{u_i u_i}}{2} \right]
\end{equation}

\subsection{Helmholtz Decomposition}
\subsubsection{For velocity field, u}
The Helmholtz decomposition theorem, or the fundamental theorem of vector
calculus,
states that any well-behaved vector field can be decomposed into
the sum of a longitudinal (diverging, non-curling, irrotational) vector field
and
a transverse (solenoidal, curling, rotational, non-diverging) vector field.
This allows us to express the velocity field as:
\begin{align}
    \label{helm_u}
    \mathbf u & = -\nabla \times (e_z \Psi) + \nabla \Phi \\
              & =  -\nabla \times \Psi_z + \nabla \Phi
\end{align}

For the sake of clarity,
we shall denote the rotational (vortical) and divergent (wave) parts of the
velocity with the suffix \emph{r} and \emph{d} respectively. Thus,
\begin{align}
    \mathbf u_r & = -\nabla \times \Psi_z ; & \mathbf u_d =   \nabla \Phi
\end{align}
And therefore, $\mathbf u  = \mathbf u_r + \mathbf u_d$.

To find the projection operators for the divergent part, taking divergence of
\eqref{helm_u} gives $\nabla .\mathbf{u} = \nabla^2 \Phi
$.
This transforms in the spectral plane as, $ik_j \hat{u}_j = -\kappa^2
    \hat{\Phi}$, implying:
\begin{align}
    \mathbf{\hat{u}}_d = & ik_i \hat{\Phi} = \frac{k_i k_j}{\kappa^2} \hat{u}_j       \\
    \mathbf{\hat{u}}_r = & \mathbf{\hat{u}} - \mathbf{\hat{u}}_d = \left( \delta_{ij}
    - \frac{k_i k_j}{\kappa^2} \right) \hat{u}_j
\end{align}

Thus, for two-dimensions the decomposed velocity are represented as follows,

\begin{align}
    \mathbf{\hat{u}}_d^x = & \frac{k_x (k_x \hat{u}_x + k_y \hat{u}_y)}{\kappa^2}                \\
    \mathbf{\hat{u}}_r =   & \hat{u}_{x} - \frac{k_x  (k_x \hat{u}_x + k_y \hat{u}_y)}{\kappa^2}
\end{align}

\subsubsection{For fluid depth, h}
To obtain a similar decomposition, $h = h_r + h_d$, one should use the Poisson
equation \eqref{poisson_fft}.
Since Poisson equation requires a divergence free flow, the LHS of
\eqref{poisson_fft} would correspond to the rotational part of the flow in the
transformed plane, i.e. $\hat{h}_r$. Henceforth, the divergent part, $\hat{h}_d$
can be obtained by subtracting $\hat{h}_r$ from $\hat{h}$.

\subsection{Normal mode decomposition}

\subsubsection{Following Bartello 1995}
In order to isolate oscillating fast modes that display no nonlinearity, one
can adopt a linearization approach followed by a normal mode decomposition.
As a result of linearization, from \eqref{dtu} and \eqref{dth}:
\begin{align}
    \label{dtu_l}
    \partial_t \mathbf u = & - c^2 \nabla \eta - f\mathbf{e_z} \times \mathbf u \\
    \label{dth_l}
    \partial_t \eta =      & - \nabla.  \mathbf u
\end{align}
Taking curl and divergence of the linearized equation for momentum
\eqref{dtu_l} gives the following evolution equations with change of variables:
\begin{align}
    \partial_t \zeta =  & - f \delta
    \label{dtcurl_l}                                                  \\
    \partial_t \delta = & f \zeta - c^2 \nabla^2 \eta \label{dtdiv_l} \\
    \partial_t \eta =   & - \delta \label{dteta_l}
\end{align}
where $\zeta$ and $\delta$ are relative vorticity and divergence as functions
of $\mathbf{r}$ and $t$ respectively. Representing the dependent flow
quantities in terms of Fourier modes:
\begin{align}
    \begin{pmatrix}
        u \\ v \\ \eta \\ \zeta \\ \delta
    \end{pmatrix} (\mathbf{r},t)
    = \int
    \begin{pmatrix}
        \hat{u} \\ \hat{v} \\ \hat{\eta} \\ \hat{\zeta} \\ \hat{\delta}
    \end{pmatrix} e^{i(\mathbf k . \mathbf{r} - \omega t)} \mathbf{dk} d\omega
\end{align}
allows to rewrite the system of equations \eqref{dtcurl_l} to \eqref{dteta_l}
as an eigenvalue problem:
\begin{align}
    i\omega
    \begin{Bmatrix}
        \hat{\zeta} \\ \hat{\delta} \\c\kappa\hat{\eta}
    \end{Bmatrix}
    = i
    \begin{bmatrix}
        0   & if       & 0         \\
        -if & 0        & -ic\kappa \\
        0   & ic\kappa & 0
    \end{bmatrix}
    \begin{Bmatrix}
        \hat{\zeta} \\ \hat{\delta} \\c\kappa\hat{\eta}
    \end{Bmatrix}
\end{align}
where, $\kappa = |\mathbf{k}|$.

Let us define $A$ as the Hermitian matrix operating on the vector
$\mathbf{W} = \{\hat{\zeta}, \hat{\delta} ,c\kappa \hat{\eta} \}^T $.which
yields the
familiar dispersion relation
for the slow geostrophic mode
and fast Poincar\'e wave modes:
\begin{equation}
    \omega^{(0)} = 0;\quad \omega^{(\pm)}=\pm \sigma
\end{equation}
where, $\sigma = \sqrt{f^2 + c^2\kappa^2 }$. These are the eigenvalues of
the matrix operator \emph{A}. Since \emph{A} is Hermitian, the corresponding
eigenvectors are orthogonal and these are normalized as follows
\begin{align}
    \mathbf X^{(0)}_n =
    \frac{1}{\sigma}
    \begin{Bmatrix}
        -c\kappa \\ 0 \\ f
    \end{Bmatrix}; \quad
    \mathbf X^{(\pm)}_n =
    \frac{1}{\sqrt{2} \sigma}
    \begin{Bmatrix}
        f \\ \mp i\sigma \\ c\kappa
    \end{Bmatrix}
\end{align}
Let $X_n$ be the eigenvector matrix, which follows the
property $X_n \bar{X}_n^{T}=I$. It can, then, be applied to diagonalize the
system of
equations as follows:
\begin{align}
    \partial_t \mathbf{W} =               & [A] \mathbf{W}                    \\
    \partial_t (\bar{X}_n^T \mathbf{W}) = & \bar{X}_n^T[A]X_n
    \bar{X}_n^T\mathbf{W}                                                     \\
    =                                     & [\Lambda] (\bar{X}_n^T\mathbf{W})
\end{align}
Thus, the alternate diagonalized system of equations for the normal modes are
given by:
\begin{align}
    \partial_t
    \mathbf{N}
    = &
    \begin{bmatrix}
        0 & 0        & 0       \\
        0 & -i\sigma & 0       \\
        0 & 0        & i\sigma
    \end{bmatrix}
    \mathbf{N}                   \\
    \text{where},
    \mathbf{N} = \bar{X}_n^T \mathbf{W}
    = & \frac{1}{\sqrt{2}\sigma}
    \begin{Bmatrix}
        -\sqrt{2}c\kappa(\hat{\zeta} -f \hat{\eta})                \\
        f\hat{\zeta} + c^2\kappa^2\hat{\eta} - i\sigma\hat{\delta} \\
        f\hat{\zeta} + c^2\kappa^2\hat{\eta} + i\sigma\hat{\delta}
    \end{Bmatrix}
\end{align}

The first mode represents \emph{linearised potential vorticity} which is
conserved in this framework. The remaining two are the \emph{linearised
    ageostrophic or wave modes}. In the absence of system rotation, i.e. $f=0$,
the normal modes are simply,
\begin{equation}
    \label{nmode}
    \mathbf{N} =
    \begin{Bmatrix}
        \hat \zeta                                              \\
        \frac{1}{\sqrt{2}} (c\kappa \hat{\eta} - i\hat{\delta}) \\
        \frac{1}{\sqrt{2}} (c\kappa \hat{\eta} + i\hat{\delta})
    \end{Bmatrix}
\end{equation}

%---------------------------------------------%
\subsubsection{Following Farge \& Sadourny 1989}
Instead of finding the normal modes for the vorticity, divergence and
displacement field of the flow, we shall make use of the Helmholtz
decomposition described in \eqref{helm_u}. The shallow water equations then
transform to:
\begin{align}
    \partial_t \psi = & f \phi
    \label{dtpsi_l}                                        \\
    \partial_t \phi = & -f \psi - c^2 \eta \label{dtphi_l} \\
    \partial_t \eta = & - \nabla^2 \phi \label{dteta_l2}
\end{align}
where $\psi$ and $\psi$ are stream function and velocity potential as functions
of $\mathbf{r}$ and $t$ respectively. By substituting the dependent
variables with the respective Fourier transform, this reduces to the eigenvalue
problem:
\begin{align}
    i\omega
    \begin{Bmatrix}
        \hat{\psi} \\ \hat{\phi} \\ \hat{\eta}
    \end{Bmatrix}
    = i
    \begin{bmatrix}
        0   & if         & 0    \\
        -if & 0          & ic^2 \\
        0   & -i\kappa^2 & 0
    \end{bmatrix}
    \begin{Bmatrix}
        \hat{\psi} \\ \hat{\phi} \\ \hat{\eta}
    \end{Bmatrix}
\end{align}
the square matrix is not Hermitian and this would result in complex
eigenvalues. By adopting the following change of variables:
\begin{equation}
    \hat{\psi} \to \kappa^2\hat \psi = \hat{\zeta} ; \quad
    \hat{\phi} \to -\kappa^2\hat \phi = \hat{\delta}; \quad
    \hat{\eta} \to c\kappa\hat \eta
\end{equation}
it falls back to the previous eigenvalue problem as demonstrated in the
previous section. In other words, we can use the same eigenvector matrix, $X_n$
to find the normal modes of:

$$\mathbf{H} = \{\hat \psi,\; \hat \phi,\;  \eta  \}^T$$
which is closely related to:
$$\mathbf{W}
    = \{\kappa^2\hat \psi,\; -\kappa^2\hat
    \phi,\; c\kappa\hat \eta  \}^T
    = \{\hat \zeta;\; \hat \delta;\;  c\kappa \hat \eta \}^T $$

%---------------------------------------------%
\subsubsection{Normal mode inversion to primitive variables}
To begin with, we shall form a new vector $\mathbf{B} = \mathbf{N}/\kappa$
which will have the same dimensions as velocity.
It has been shown before that the normal modes can be represented by $\mathbf{N}
    = \bar{X}_n^T \mathbf{W}$. Now,  we shall form a new vector $\mathbf{B} =
    \mathbf{N}/\kappa$ which will have the same dimensions as velocity. Thus,
\begin{align}
    \mathbf{B}
    = & \frac{1}{\sqrt{2}\sigma}
    \begin{Bmatrix} \sqrt{2} c\left(-
        \kappa^{2} \hat \psi +  f\hat\eta \right)               \\
        \kappa \left(c^{2} \eta + f \psi + i \phi \sigma\right) \\
        \kappa \left(c^{2} \eta + f \psi - i \phi \sigma\right)
    \end{Bmatrix}
\end{align}

The $\mathbf{B}$ vector can also be related to the primitive variable vector,
$\mathbf{U} = \{\hat{u},\hat{v},c\hat{\eta}\}^T$ using transformation matrices
as follows:
\begin{align}
    \mathbf{B} = & \frac{1}{\kappa}[\bar{X}_n]^T W              \\
    =            & \frac{1}{\kappa}[\bar{X}_n]^T [P] \mathbf{U}
\end{align}
where,
\begin{equation}P=
    \begin{bmatrix}- i k_{y} & i k_{x} & 0 \\ i k_{x} &  i
        k_{y}     &
        0                       \\0 & 0 & \kappa\end{bmatrix}
\end{equation}
It is also straightforward to show that the magnitude of the normal modes
equals the total energy as:
\begin{align}
    \mathbf{\bar B}^T\mathbf{B}
    = & \frac{1}{\kappa^2} \mathbf{\bar U}^T
    [\bar P]^T[{X}_n]  [\bar{X}_n]^T [P] \mathbf{U} \\
    = & \mathbf{\bar U}^T\mathbf{U}
\end{align}
implying, $$2(E_K+E_P)=\sum_i B^{(i)}\bar{B}^{(i)}=\kappa^{2} \psi
    \overline{\psi} +\kappa^{2} \phi \overline{\phi} + c^{2}
    \eta \overline{\eta}
    = u\bar{u} + v\bar{v} + c^2\eta\bar{\eta}$$
This allows us to represent the primitive variables in terms of normal modes as
$$\mathbf{U} = [Q]\mathbf{B}$$
where, the inversion matrix is
\begin{align}
    Q
    = & \kappa [P]^{-1}[\bar{X}_n]^{T^{-1}} \\
    = & \frac{1}{\sqrt{2}\sigma\kappa}
    \begin{bmatrix}
        -i\sqrt{2}c\kappa  k_{y}                    &
        \left(i f k_{y} +
        k_{x}   \sigma\right)                       &
        \left(i f k_{y} - k_{x} \sigma\right)         \\
        i \sqrt{2}c\kappa k_{x}                     &
        \left(-  i f k_{x} +  k_{y}   \sigma\right) &
        - \left(i f  k_{x} + k_{y} \sigma\right)      \\
        \sqrt{2}\kappa f                            &
        c\kappa^{2}                                 &
        c\kappa^{2}
    \end{bmatrix}
\end{align}
In tensor notation,
\begin{align}
    \hat{u}_l =   & \epsilon_{lm3} ik_m\left[ -\frac{c}{\sigma} B^{(0)} +
        \frac{f}{\sqrt{2}\sigma\kappa} (B^{(+)} + B^{(-)})
        \right]
    + k_l \frac{1}{\sqrt{2}\kappa}  (B^{(+)} - B^{(-)})                       \\
    c\hat{\eta} = & \frac{f}{\sigma} B^{(0)} + \frac{c\kappa}{\sqrt{2}\sigma}
    (B^{(+)} + B^{(-)})
\end{align}

\subsection{Spectral energy budget}
\subsubsection{Kinetic energy}
%---------------------------------------------%
\paragraphbf{Expression for mean kinetic energy}
Kinetic energy being cubic for shallow water equations, it can be split into
quadratic and non-quadratic parts as follows:
\begin{align*}
    E_K(\mathbf{r},t)
    = & h\mathbf{u}.\mathbf{u}                            \\
    = & (1+\eta)\mathbf{u}.\mathbf{u}                     \\
    = & \mathbf{u}.\mathbf{u} + \eta\mathbf{u}.\mathbf{u} \\
    = & E_K^Q + E_K^{NQ}
\end{align*}

%---------------------------------------------%
\paragraphbf{K.E. budget}
Let the rate of change of kinetic energy, without any approximations, be
written as:
\begin{equation}\label{dtKE}
    \pder{t}E_K(\mathbf{k},t) = T_K + C_K
\end{equation}
where, $T_K$ and $C_K$ represents the transfer and conversion spectral
functions respectively. Equation
\eqref{dtKE} splits into:
\begin{align*}
    \pder{t}E_K^Q + \pder{t}E_K^{NQ}
    = & (T_K^{Q} + C_K^{Q}) + (T_K^{NQ} + C_K^{NQ})
\end{align*}
%-------------------------------------------------%
\paragraph{Quadratic K.E. budget:}
Consider the rate of change of quadratic kinetic energy,
\begin{equation}\label{dtKE2}
    \pder{t}E_K^{Q}(\mathbf{k},t) = T_K^{Q} + C_K^{Q}
\end{equation}
Following \eqref{dtu0}, equation \eqref{dtKE2} expands into:
\begin{align*}
    \partial_t E_K^{Q}(\mathbf{k},t)
    = & \frac{1}{2}\partial_t(\mathbf{\hat u}.\mathbf{\hat u^*})       \\
    = & \frac{1}{2}\left[ \mathbf{\hat u} .\pder[\mathbf{\hat u^*}]{t}
        + \mathbf{\hat u^*}. \pder[\mathbf{\hat u}]{t}\right]          \\
    = & \frac{1}{2}\left[ -\hat{u}_i (\widehat{ u_j\partial_j u_i })^*
        - c^2 \hat{u}_i (ik_i \hat{\eta})^*
        - \hat{u}_i (\epsilon_{i3k} f \hat{u}_k)^*
        - ... \text{conjugate terms}
        \right]                                                        \\
    = & -\text{Re}\left[ \hat{u}_i (\widehat{ u_j\partial_j u_i })^*
        + c^2 \hat{u}_i (ik_i \hat{\eta})^*
        + \hat{u}_i (\epsilon_{i3k} f \hat{u}_k)^* \right]
\end{align*}
% since, $u_i (\epsilon_{ijk} \zeta_j u_k) = 0$
% \footnote{ \textbf{Note}
% $\epsilon_{ijk} \zeta_j u_k 
% = \epsilon_{jki}  (\epsilon_{jlm} \partial_l u_m +   f_j )u_k$}
% =& (\delta_{kl}\delta_{im} - \delta_{km} \delta_{il})\partial_l u_m +    
% \epsilon_{ijk} f_j u_k\\
% =& u_k\pder{x_k}u_i - \pder{x_i}u_ku_k + \epsilon_{ijk} f_j u_k
we have,
\begin{align}
    T_K^{Q}= & -\text{Re}\left[\hat{u}_i (\widehat{ u_j\partial_j u_i })^*
        \right]                                                              \\
    C_K^{Q}= & -\text{Re}\left[   c^2 \hat{u}_i (ik_i \hat{\eta})^*  \right] \\
    =        & -\text{Re}\left[   c^2 \hat{u}_i^* ik_i \hat{\eta} \right]
\end{align}

%-------------------------------------------------%
\paragraph{Non-quadratic K.E. budget:}
\begin{align*}
    \pder{t}E_K^{NQ}(\mathbf{k},t) = & T_K^{NQ} + C_K^{NQ}
\end{align*}
Following equations \eqref{dtu0} and \eqref{dtM} for rate of change of velocity
$\mathbf{u}$ and the displaced mass flux, $\mathbf M = \eta \mathbf{u}$
respectively,
\begin{align*}
    \partial_t E_K^{NQ}(\mathbf{k},t)
    = & \frac{1}{2}\partial_t(\mathbf{\hat M}.\mathbf{\hat u^*})        \\
    = & \frac{1}{2}\left[ \mathbf{\hat M} .\pder[\mathbf{\hat u^*}]{t}
        + \mathbf{\hat u^*}. \pder[\mathbf{\hat M}]{t}\right]           \\
    = & \frac{1}{2}\left[-\hat{M}_i(\widehat{ u_j\partial_j u_i })^*
        - c^2 \hat{M}_i (ik_i \hat{\eta})^*
        - \hat{M}_i (\epsilon_{i3k} f \hat{u}_k)^* \right]              \\
      & +\frac{1}{2}\left[ - \hat{u}_i^* \widehat{(u_j\partial_j M_i) }
        - c^2\hat{u}_i^*ik_i\widehat{\eta\eta}/2
        - \hat{u}_i^* (\epsilon_{i3k} f \hat{M}_k)
        - \hat{u_i}^* \widehat{(u_i \partial_j({u}_j+
            M_j))}
        \right]
\end{align*}
Hence,
\begin{align}
    T_K^{NQ}=       & -\frac{1}{2}\left[\hat{M}_i(\widehat{ u_j\partial_j u_i })^*
        + \hat{u}_i^* \widehat{(u_j\partial_j M_i) }
        + \hat{u_i}^* \widehat{(u_i \partial_j{M}_j)}
        \right]                                                                    \\
    T_{K,K}^{NQ,Q}= & -\frac{1}{2}\left[ \hat{u_i}^* \widehat{(u_i
            \partial_j{u}_j)}
        \right]                                                                    \\
    T_{K,P}^{NQ,Q}= & -\frac{1}{2}\left[c^2 \hat{M}_i (ik_i \hat{\eta})^*
        + c^2\hat{u}_i^*ik_i\widehat{\eta\eta} /2
        \right]
\end{align}

\paragraphbf{Helmholtz decomposition of kinetic energy}

Furthermore, using Helmholtz decomposition to substitute for $\mathbf u$ in
the expression
for \emph{mean kinetic energy},

\begin{align}
    \braket{ E_K}
     & = \frac{1}{2} \braket{ h\mathbf u.\mathbf u}                          \\
     & = \frac{1}{2} \braket{h(\mathbf{u_r} + \mathbf{u_d} ).(\mathbf{u_r} +
        \mathbf{u_d} )}                                                      \\
     & = \frac{1}{2}\braket{ h |\mathbf{u_r} | ^2}
    -\braket{ \mathbf{u_r}.h\mathbf{u_d}}
    +  \frac{1}{2} \braket{ h | \mathbf{u_d}| ^2}
\end{align}
where, $h = 1 + \eta$ and, $\eta$ is the surface displacement of the shallow
water layer. In the limit for small displacements i.e., $\eta \to 0$ or $ h \to
    1$, invoking orthogonality,

\begin{equation}
    \lim_{h \to 1} \braket{E_K} = \braket{E_K^{Q}}
    =  \frac{1}{2}\braket{|\mathbf{u_r}  | ^2 + | \mathbf{u_d} | ^2}
\end{equation}
The spectral equivalent of the above expression will be:
\begin{equation}
    \braket{E_K^{Q} }
    =  \frac{1}{2}\braket{ \mathbf{\hat{u}_r}.\mathbf{\hat{u}_r}^* +
        \mathbf{\hat{u}_d}.\mathbf{\hat{u}_d}^* }
\end{equation}

%---------------------------------------------------------%
\subsubsection{Available potential energy}
\paragraphbf{Expression for mean available potential energy}
\begin{align}
    \braket{ E_P}
     & = \frac{c^2}{2} \braket{ \eta^2}
\end{align}
The spectral equivalent of which is:
\begin{align}
    \braket{ E_P}
     & = \frac{c^2}{2} \braket{ \hat{\eta}\hat{\eta}^*}
\end{align}

\paragraphbf{A.P.E. budget}
Similar to equation \eqref{dtKE} we can form,
\begin{equation}\label{dtPE}
    \pder{t}E_P(\mathbf{k},t) = T_P+ C_P
\end{equation}
where, $T_P$ and $C_P$ represents the transfer and conversion spectral
functions respectively.

Following \eqref{dth}, equation \eqref{dtPE} expands as:
\begin{align*}
    \pder{t}E_P(\mathbf{k},t)
    = & \frac{c^2}{2}\partial_t({\hat \eta}{\hat \eta^*})                 \\
    = & \frac{c^2}{2}\left[ \hat{\eta} .\pder[\hat{\eta}^*]{t}+ \hat
        \eta^*.\pder[\hat \eta]{t}\right]                                 \\
    = & \frac{c^2}{2}\left[ - \hat{\eta} (ik_i\hat{ u_i } + ik_i\widehat{
            \eta u_i })^*
        - \hat{\eta}^* (ik_i\hat{ u_i } + ik_i\widehat{
            \eta u_i })  \right]                                          \\
    = & -\text{Re}\left[c^2\hat{\eta} (ik_i\hat{ u_i } + ik_i\widehat{
            \eta u_i })^* \right]
\end{align*}
Thus,
\begin{align}
    T_P= & -\text{Re}\left[c^2\hat{\eta} (ik_i\widehat{
            \eta u_i })^*  \right]                                 \\
    C_P= & -\text{Re}\left[c^2\hat{\eta} (ik_i\hat{u_i })^*\right] \\
    =    & +\text{Re}\left[c^2\hat{\eta} ik_i\hat{u_i }^*\right]
\end{align}
thus, $C^Q_K = -C_P$ which, in turn, makes the assertion that
equivalent conversion occurs between quadratic K.E. and A.P.E.
% 
% \begin{align*}
% 2\times T_K(\mathbf{k},t)
% & = \partial_t (\mathbf u.h\mathbf u)\\
% &= \partial_t \mathbf u .h\mathbf u  +|\mathbf u|^2 \partial_t h + h\mathbf u 
% .\partial_t \mathbf u \\
% & =   2 \left( -  \frac{\nabla|\mathbf u|^2}{2} - c^2 \nabla h - \zeta_a \times \mathbf u	\right).h\mathbf u - \left( |\mathbf u|^2 \nabla.(h \mathbf u )\right)\\ 
% & =   \left( -  h\frac{\nabla|\mathbf u|^3}{3} - c^2 \frac{\nabla h^2}{2} . \mathbf u \right)- \left( |\mathbf u|^2 \nabla.(h \mathbf u )\right) \\
% \end{align*}
% 
% 

% \end{document}
